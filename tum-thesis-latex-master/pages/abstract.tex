\chapter{\abstractname}
Invasive computing is a novel paradigm for the design and resource-aware programming of future parallel computing systems. It enables the programmer to write resource aware programs and the goal is to optimize the program for the available resources. Traditionally, parallel applications implemented using MPI are executed with a fixed number of MPI processes before submitting to a HPC(High Performance Computing) system. This results in a fixed allocation of resources for the job. Newer techniques in scientific computing such as AMR(Adaptive Mesh Refinement) result in applications exhibiting complex behavior where their resource requirements change during execution. Invasive MPI which is a part of an ongoing research effort to provide MPI extensions for the development of Invasive MPI applications will result in evolving jobs for the HPC systems during runtime that utilize such AMR techniques. Unfortunately, using only static allocations result in the evolving applications being forced to execute using their maximum resource requirements that may lead to an inefficient resource utilisation. In order to support such parallel evolving applications at HPC centers there is an urgent need to investigate and implement extensions to existing resource management systems or develop an entirely new one. This thesis will extend the work done over the last few months during which an early prototype was implemented by developing a protocol for the intgeration of invasive resource management into existing standard batch systems. Specifically, This thesis will now investigate and implement a job scheduling algorithm in accordance with the new protocol developed earlier for supporting such an invasive resource management.\par

%TODO: Abstract


