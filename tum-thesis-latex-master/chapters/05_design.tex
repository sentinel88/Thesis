\chapter{Design}
In this section, we will describe the internal details of the components in the architecture specifically on how the scheduling algorithms work in order to realize the negotiation, the meaning of job mappings, resource offers and feedback reports. 
\label{chapter:ischeduler}
\section{Entities}
A design entity is an element (component) of a design that is structurally and functionally distinct from other elements and that is separately named and referenced. They result from the decomposition of the software system requirements. The objective is to divide the system into separate components that can be considered, implemented, changed, and tested with minimal effect on other entities. Examples of design entities are: \textbf{\textit{protocol, application layer, state machine, data model, object, task, sub-systems, modules}} etc.\\ \\
\textbf{Job Mappings}\\
A job mapping is basically a job and its set of allocated resources. A forward job record is a job mapping and other related job details important for its launch. A list of such forward job records are sent by the batch scheduler in response to receiving a resource offer from the runtime scheduler.\\
%\vspace{-0.5in}
%\begin{center}
\begin{equation*}
%\begin{multlined}
\begin{aligned}
&Map\ \ \ \textit{:=\ \ \ {$<Map_{1}><Map_{2}>...<Map_{n}>$}}\\
&Map_{i}\ \ \ \textit{:=\ \ \ Job Record\ \ \ :=}\ \ \ \{M_{i},details_{i}\}\\
&M_{i}\ \ \ \textit{:=\ \ \ Job Mapping\ \ \ :=}\ \ \ \{jobid,node count\}
%\end{multlined}
\end{aligned}
\end{equation*}
%\end{center}
In the above description $details_{i}$ refers to the details of any $i^{th}$ job such as \textit{constraints, priority, runtime estimate, max nodes, min nodes, job name, start time, deadline}. In these details, the constraints are specific to the purpose of negotiation and this can be related to node count that the job is requesting for, memory size, exclusive / shared resource access etc. Currently, we support only the constraints for node count. The node count constraint will again have a min and max both of which will fall within the window of max and min nodes of the job. It is using these constraints that the batch scheduler will negotiate with the runtime scheduler and with every negotiation attempt it will relax the constraints to bargain better.\\ \\
\textbf{Resource Offers}\\
A reservation is basically a job being guaranteed a set of resources for a certain period of time. A resource offer is a list of such reservations and the set of nodes which are free in the partition after the runtime scheduler has guaranteed these reservations. The runtime scheduler will send such a resource offer to the batch scheduler if it rejected its mapping or it is sending a new offer that it generated.\\
%\vspace{-0.5in}
%\begin{center}
\begin{equation*}
%\begin{multlined}
\begin{aligned}
&Offer\ \ \ \textit{:=\ \ \ $<Resv_{1}><Resv_{2}>...<Resv_{n}>$,\ IdleNodes}\\
&Resv_{i}\ \ \ \textit{:=\ \ \ Job Reservation\ \ \ :=}\ \ \ \{jobid,node count,start\ time,end\ time\}\\
&IdleNodes\ \ \ \textit{:=\ \ \ Free Nodes\ \ \ :=}\ \ \ \{nodecount\}
%\end{multlined}
\end{aligned}
\end{equation*}
%\end{center}
The \textbf{\textit{$Resv_{i}$}} entry in the resource offer corresponding to a particular job id must match the job id of the \textbf{\textit{$Map_{i}$}} entry in the forwarded jobs. This means that the sequence in which jobs were forwarded to the runtime scheduler will be the same sequence of jobs for which reservations will be sent to the batch scheduler.\\ \\
%%%%%%%%%
\textbf{Feedback Reports}\\
Periodic or event-driven feedback reports on the latest status of the running jobs are sent by the runtime scheduler to the batch scheduler. It is event-driven for events such as job termination, job completion, acceptance of a mapping received from batch scheduler.
%\vspace{-0.2in}
%\begin{center}
\begin{equation*}
%\begin{multlined}
\begin{aligned}
&Feedback\ \ \ \textit{:=\ \ \ $<Report_{1}><Report_{2}>...<Report_{n}>$}\\
&Report_{i}\ \ \ \textit{:=\ \ \ Status Report\ \ \ :=}\ \ \ \{jobid,node count,runtime,end\ time,state,P_{i}\}\\
&P_{i}\ \ \ \textit{:=\ \ \ Performance Data\ \ \ :=}\ \ \ \{model,energy,io,memory,hint\}
%\end{multlined}
\end{aligned}
\end{equation*}
The feedback reports can also contain performance specific data such as energy consumption, performance model of completed jobs that can be used by the batch scheduler for making better decisions on same jobs submitted again in the future. For running jobs, the reports can contain similar information but most important ones would be the current assigned node count, expected end time, memory usage etc.
%\end{center}
%%%%%%%%%%
\section{Scheduling Algorithms}
Following pages present the pseudo code for both the batch and runtime scheduling algorithms.
%%%%%%%%%
\subsection{Batch Scheduling}
%\clearpage
%%%%%%
%%%%%%
%\RestyleAlgo{boxed}
\RestyleAlgo{boxruled}
\IncMargin{1em}
\begin{algorithm}[!htbp]
 \DontPrintSemicolon
 \KwIn{Resource Offer}
 \KwOut{Map : Jobs $\rightarrow$ Offer}
 \tcc{Invasive job queue consists of jobs submitted specifically to the invasic partition}
 build invasive job queue\;
 \If{empty queue}{
  return empty Map\;
 }
 sort queue by job priorities\;
 reservations $\leftarrow$ Offer.Reservations\;
 freenodes $\leftarrow$ Offer.ResidualNodes\;
 \While{not at end of the invasive job queue}{
  read next queue entry\;
  \If{entry.jobId in reservations}{
   resv $\leftarrow$ reservations(entry.jobId)\;
   \eIf{resv.node\_count EQ 0}{
    entry.start\_time $\leftarrow$ resv.start\_time\;
    entry.node\_count $\leftarrow$ $0$\;
    adjustNodeCount(entry.jobId)\;
   }{
    entry.start\_time $\leftarrow$ $0$\;
    entry.node\_count $\leftarrow$ resv.node\_count\;
    create forward job record using entry\;
    append record to Map\;
    mapped $\leftarrow$ true\;
   }
  }
  \If{mapped NEQ true}{
   try mapping to the freenodes\;
   \If{successfully mapped}{
    update the count of freenodes\;
   }
  }
  create forward job record using entry\;
  append record to Map\; 
 }
 try\_to\_best\_fit(Map, invasive job queue, Offer)\;
 return Map\;
 \caption{Batch Scheduling Algorithm}
 \label{algo:1}
\end{algorithm}
%%%%%%%
\textbf{\textit{Algorithm }}\ref{algo:1} presents a high level pseudo code of the batch scheduling algorithm. Following points describe the algorithm:
\begin{itemize}
\item It takes as input a resource offer from iRTSched and decides whether to accept or reject it. Irrespective of the decision, the batch scheduler has to send a list of jobs to the runtime scheduler. 
\item In this list some or all of the jobs will be mapped to the resources offered and some of them may not have any mapping but will just be forwarded along to respect the order of jobs in the queue. This allows for \textbf{\textit{fairness}} and \textbf{\textit{avoids starvation}} when other small jobs may potentially keep overtaking these waiting jobs in getting mapped.
\item We will refer to jobs which do not get mapped as a \textbf{\textit{reserved job}} for our convenience whereas those which have resources allocated or have been mapped are called as \textbf{\textit{mapped jobs}}. The mapped jobs and reserved jobs together form a list of jobs to be forwarded.  
\item At a high level, the algorithm can be viewed as a sequence of $3$ important steps that will be described in the following points along with references to where they are located in the algorithm.
\item \textbf{\textit{Line 1-33}}: basically represents the \textbf{\textit{step}} $1$ of the algorithm. \textbf{\textit{Line \boldmath{$1$}}}: A separate job queue is constructed by scanning the main job queue for jobs which have been submitted specifically to the invasic partition. These jobs are then sorted according to their priorities.
\item \textbf{\textit{Line \boldmath{$8$-$34$}}}: For every job in the invasive queue: If it is found in the reservation list from the resource offer and if it can start immediately(has a node count allocated) then we will just append this to a new list (Map) after creating a new forward job record using the details of this job.
\item If it starts in the future (has a zero node count) then we will relax its node count constraints by a step size calculated as below and try to map it to the freenodes provided in the offer.
\vspace{-0.30in}
\begin{center}
\boldmath\begin{equation*}
step\ size = \frac{(Job.MaxNodes - Job.MinNodes)}{MAX\_NEGOTIATION\_ATTEMPTS}
\end{equation*}
\end{center}
\item If \textbf{\textit{step size}} in the above calculation comes up as $0$ then we will set it as $1$.
\item \textbf{\textit{job.node\_count.min $=$ job.node\_count.min - step size}}. If the computed value is negative or less than the job's minimum number of nodes then we just set \textbf{\textit{job.node\_count.min $=$ job.min\_nodes}}.
\item If the job is not found in the reservation list then this must be a new job that was submitted to the batch scheduler in between the negotiations and was not considered during the previous negotiation attempt. And, this job has been placed near the front of the queue owing to high priority or because it is a small job etc. We will directly try to map it to the freenodes in the offer.
\item If a job could not be mapped, then we will just append it to the Map by creating a new forward job record. If it was mapped, then we will set the current node count of that job to the count of the allocated nodes. The node count constraints remain the same.
\item \textbf{\textit{Line \boldmath{$34$}}}: This routine basically represents the \textbf{\textit{step}} $2$ and $3$ of this algorithm. The result of these two steps is an approxmiation of a best fit mapping of the invasic jobs ready to be forwarded to iRTSched.
\item \textbf{\textit{Algorithm }}\ref{algo:2} This algorithm shows how the best fit mapping is computed. \textbf{\textit{Line 8}}: We will try to map reserved jobs from the newly constructed job list (Map) by reducing the node counts of jobs that have been currently mapped within their constraints. This operation considers mapped jobs in the reverse order of their priority so that the operation can start from the lowest priority mapped job till the highest one. 
\item If we were successful in mapping any new reserved jobs by the above operations, then we must rescan the invasive job queue to update the new job list by jobs which had been previously skipped in \textbf{\textit{step}} $1$ due to a limit on the \textbf{\textit{reservation depth}}. Reservation depth means that we have a limit on the number of reserved jobs that can be forwarded to the runtime scheduler. This is the \textbf{\textit{step 3}} of the algorithm.
\item \textbf{\textit{Algorithm }}\ref{algo:3} This algorithm shows how the analysis of mapped jobs are done in order to reduce their allocated node counts to satisfy a reserved job.
\vspace{-0.30in}
\begin{center}
\boldmath\begin{equation*}
step\ size = \frac{(Job.node\_cnt - Job.node\_count.min)}{MAX\_NEGOTIATION\_ATTEMPTS}
\end{equation*}
\end{center}
\end{itemize}
Batch scheduler will provide \textbf{\textit{fairness}} to jobs while scheduling by considering them in the order of decreasing priorities. It also avoids any \textbf{\textit{job starvation}} by forwarding jobs that could not get mapped to the available resource offer. This will result in the runtime scheduler giving these jobs future start times as per its backfill algorithm and only then process other job requests after it  in the list of forwarded jobs. Batch scheduler will also adjust the expected end times of those jobs that it will transform by expand / shrink. This adjustment will take into consideration the performance model of the running application.\\ \\
%%%%%%
\textbf{\textit{Decision Logic}} The batch scheduler currently has a very simple decision making routine. It tries to maximize the sum of the priorities(objective) of jobs that have been mapped to some resources in the offer. Below are the key points:
\begin{itemize}
\item Batch scheduler will keep sending a reject if the job at the front of the queue could not be mapped or atleast some fraction of the total number of jobs in the queue have not been mapped(ex: one-third of the total jobs). But, if the number of negotiations have crossed half the limit, then it will ignore these constraints.
\item If the batch scheduler had sent an accept decision on the previous negotiation attempt, then it will send an accept again in the current attempt if the objective value has remained the same or improved. It will also stop relaxing the node count constraints of jobs for further negotiations in case the number of jobs which have been mapped are more than the previous attempt.
\item If the batch scheduler had sent a reject on the previous negotiation attempt, then it will send a reject again in the current attempt if the objective has reduced. If the objective has remained the same or improved, then it will accept if it does not violate the first point.
\end{itemize}
%%%%%
\setcounter{AlgoLine}{0}
%%%%%%%
%\pagebreak
%\newpage
\begin{algorithm}[!htbp]
 \DontPrintSemicolon
 \KwIn{Map, Invasive Job Queue, Offer}
 \KwOut{Updated Map : Jobs $\rightarrow$ Offer}
 avail\_bitmap $\leftarrow$ Offer.ResidualNodes\;
 \While{not at end of the Map}{
  read Map entry\;
  \If{(entry.start\_time EQ $0$) AND (entry.bitmap NEQ NULL)}{
   continue\;
  } 
  \tcc{Try to map this job by shrinking other mapped jobs}
  try\_sched(entry, Map, avail\_bitmap)\;
  \If{successfully scheduled}{
   update avail\_bitmap\;
  }
 }
 \If{avail\_bitmap not empty}{
  Rescan the invasive job queue to fill the residual nodes with some new jobs\;
 }
 \caption{Best Fit Algorithm}
 \label{algo:2}
\end{algorithm}
\setcounter{AlgoLine}{0}
%%%%%%%%%
%\IncMargin{1em}
\begin{algorithm}[!htbp]
 \DontPrintSemicolon
 \KwIn{Entry, Map, avail\_bitmap}
 \KwOut{Updated entry: Mapped to some nodes}
 \tcc{The mapped jobs in the Map would be analyzed in the reverse order which is increasing in the priority}
 Analyze in increasing order of priority if the mapped jobs in the Map can be shrunk to find enough nodes for entry\;
 \If{sufficient nodes available}{
  shrink the mapped jobs as per the analysis\;
 }
 entry.bitmap $\leftarrow$ bitmap(available nodes)\;
 \caption{Try Schedule Algorithm}
 \label{algo:3}
\end{algorithm}
\setcounter{AlgoLine}{0}
%%%%%%%
\subsection{Runtime Scheduling}
\begin{algorithm}[!t]
 \DontPrintSemicolon
 \KwIn{Attempts, Jobs2Map, Error Code, Offer<empty>}
 \KwOut{Offer<Reservations, Residual nodes>}
 \If{Error Code is SUCCESS}{
 \tcc{Batch Scheduler accepted the previously sent offer}
  \If{(Jobs2Map NEQ NULL) AND (Attempts GT $1$)}{
   initialize runtime state\;
  }
  \tcc{Repeat the transformation of the running jobs which was done for the previous negotiation attempt}
  schedule((Attempts GT $1$) ? (Attempts - 1) : Attempts, Jobs2Map, Error Code, Offer)\;
  \tcc{Offer being generated for the first time if attempts is equal to $1$}
  \If{Attempts EQ $1$}{
   return SUCCESS\;
  }
  \If{schedule was successful AND Jobs2Map NEQ NULL AND Attempts GT $1$}{
   commit the mapped jobs to the running list\;
   return SUCCESS\;
  }
 }
 \tcc{Batch Scheduler rejected the previously sent offer}
 \If{Error Code NEQ SUCCESS}{
  initialize runtime state\;
  schedule(Attempts, Jobs2Map, Error Code, Offer)\;
 }
 return error code\; 
 \caption{Algorithm for generating a resource offer}
 \label{algo:4}
\end{algorithm}
\noindent
\textbf{\textit{Algorithm }}\ref{algo:4} presents a high level pseudo code of the algorithm to generate a resource offer. Following points describe the algorithm:
\begin{itemize}
\item If batch scheduler accepted the previously sent offer, then we will repeat the system transformation of the previous negotiation attempt to check if runtime scheduler will accept this mapping. If the mapping was accepted, then the runtime scheduler will commit the mapped jobs to the running list. 
\item If this is the first attempt and there are no forwarded jobs then it means that the runtime scheduler is generating a new offer. In this case it will perform a partial transformation and will return back from the function to send the offer.
\item If the mapping was rejected after repeating the previous transformation, then the runtime scheduler will reset the runtime state of the system and perform a new transformation for generating an offer to satisfy the forwarded job requests.
\item If the response from batch scheduler was reject, then the runtime scheduler will go through the same step as described in the previous point.
\end{itemize}
\textbf{\textit{Algorithm }}\ref{algo:5} presents a high level pseudo code of the runtime scheduling algorithm. The runtime scheduling algorithm in this thesis is based on the \textbf{\textit{DBES}} algorithm\cite{laxmikant}. It has been adapted for negotiation based scheduling in this work. Following points describe the algorithm:
\begin{itemize}
\item \textbf{\textit{Line 1:}} Run the backfill algorithm to schedule the forwarded job requests. Some jobs can immediately start whereas for others it will give reservations. Both of these are written into the resource offer.
\item \textbf{\textit{Line 2:}} If there were no forwarded job requests then this was a new offer being generated.
\item \textbf{\textit{Line 5:}} Update job dependencies according to the new system state. This means that some running malleable jobs may be maintaining invalid job dependencies. If a forwarded job can immediately start and a running malleable job has a dependency from the same job id, then this dependency should be cleared now as it is no longer valid. 
\item \textbf{\textit{Line 6-9:}} For every reserved job, Analyze running malleabe jobs according to the approach mentioned in the pseudo code for shrink operations to generate sufficient resources for immediate start of reserved jobs. If analysis was successful, then shrink the jobs.
\item \textbf{\textit{Line 14:}} Reschedule forwarded job requests using backfill algorithm and compute reservations(without job start). In this step we will not consider jobs which could get immediate start time in the previous steps. Update the resource offer accordingly.
\item \textbf{\textit{Line 15:}} Update job dependencies according to the new system state similar to point $3$.
\item \textbf{\textit{Line 16-20:}} For every reserved job, if it has a dependency on a running malleable job then we will expand that running job to the maximum possible node count and update the dependency on that running malleable with the job id of this forwarded job. Expansion of the running job will also consider shrinking other running jobs for its needs. For the purpose of shrinking, it will follow the same approach as shown in \textbf{\textit{Line $7$}} except that it will follow only the second and third steps.
\item \textbf{\textit{Line 25:}} Once all the previous steps of the algorithm have been completed, runtime scheduler will finally decide whether to accept or reject the mapping based on some decision making logic. If the number of negotiation attempts have reached the limit then it will just accept. If it rejects then it will send back the constructed offer to iBSched.
\end{itemize}
Runtime scheduler can generate a resource offer for the following events: \textbf{\textit{job termination, job completion, request for a resource offer, periodic runtime transformations of the running jobs resulting in free resources}}.\\ \\
%%%%%%%
A resource offer is sent to the batch scheduler on every alternate periodic time step. This is just a very simple \textbf{\textit{fairness}} policy to equally favor the running jobs and batch jobs. Between any two transactions, the runtime scheduler will perform a transformation of running jobs based on their peformance and also sometimes based on their dependencies (due to the DBES algorithm). During the other periodic time steps, The runtime scheduler will initiate a transaction with batch scheduler by creating an offer and sending it. It will do this by performing a partial transformation which means that it will shrink those running jobs which are shrinkable based on their performance but not use the free resources to expand other jobs and instead use it to send an offer. These partial transformations will only be done on every alternate periodic time step in order to favor running jobs and batch jobs equally. Sending an offer at every periodic time step may degrade the performance of running applications by driving them towards their minimum node counts whereas sending an offer too infrequently may effect throughput and quality of service to batch scheduler.\\ \\
%%%%%%%%%
\textbf{\textit{Runtime Transformation}} Due to the usage of DBES algorithm, a running malleable job always keeps track of the job id of a batch job that did not start till now but had its dependency on this running job. During the periodic runtime transformations (not partial), if there are any running malleable jobs which have such dependencies then they will be given complete preference for expansion at the cost of shrinking all other jobs without any dependencies. In case there are no such running jobs with dependencies then transformation will be done solely on the basis of performance and scalability.
%%%%%%
\clearpage
\setcounter{AlgoLine}{0}
%%%%%%%
%\pagebreak
%\newpage
\begin{algorithm}[H]
 \DontPrintSemicolon
 %\TitleOfAlgo{Runtime Scheduling Algorithm}
 \KwIn{Attempts, Jobs2Map, Error Code, Offer}
 \KwOut{Updated Offer}
 schedule requests and create reservations(without job start)\;
 \If{Jobs2Map EQ NULL}{ 
  return  \tcc*[h]{new resouce offer generated}\;
 }
 update job dependencies according to the new system state\;
 \For{each reserved job}{ \do 
  Prioritize malleable jobs in the order: (1) malleable job expanded for this reserved job, (2) malleable job expanded for no specific reserved  job, (3) malleable job expanded for other reserved jobs\;
  Analyze if expanded malleable jobs can be shrunk in the above order to make enough nodes available to start the reserved job\;
  \If{enough nodes found then}{
   Shrink the selected malleable jobs\;
   Insert the job as an entry in Offer.Reservations\;
  }
 }
 reschedule requests and create reservations(without job start)\;
 update job dependencies according to the new system state\;
 \For{each reserved job}{ \do 
  \If{job depends on a malleable job}{
   Expand the malleable job with the available nodes\;
  }
 }
 \If{Error Code is SUCCESS}{ \tcc{Runtime scheduler will decide now whether to accept / reject the mapping}
  \If{Attempts LT MAX\_LIMIT}{
   decision $\leftarrow$ decision\_logic(Offer, Jobs2Map, Attempts, count(idle nodes))\;
  }
 }
 \If{Attempts EQ MAX\_LIMIT}{
  decision $\leftarrow$ accept\;
 }
 \caption{Runtime Scheduling Algorithm}
 \label{algo:5}
\end{algorithm}
\RestyleAlgo{boxed}
\pagebreak
%%%%%%%%
\begin{algorithm}[!t]
 \DontPrintSemicolon
 \If{Error Code is SUCCESS AND decision is reject}{
  return\;
 }
 equipartition the available idle nodes among other remaining running malleable jobs\;
 return from the function with newly constructed resource offer the job mapping received\;
 %\caption{Runtime Scheduling Algorithm}
\end{algorithm}
\RestyleAlgo{boxruled}
%%%%%%
\noindent
\textbf{\textit{Decision Logic:}} Runtime scheduler like batch scheduler uses a very simple decision making routine. This is due to the fact that it does not manage any resources now but only simulates it for the early prototype developed in this thesis. Below are the rules it follows to accept or reject a mapping:
\begin{itemize}
\item If there are any idle nodes left after the complete runtime scheduling algorithm has run, then the received mapping will be rejected.
\item If only one job could be immediately started out of the many that have been forwarded, then it will reject the mapping else accept. This
constraint will be ignore, however, if the number of negotiations have crossed half the limit.
\item If it rejected on any negotiation attempt, then on subsequent attempts it will keep rejecting if the number of jobs immediately starting does not improve.
\item If the batch scheduler had rejected the previous offer but accepted the current offer, then runtime scheduler will again employ the same rules for making a decision as enumerated in the last three points.
\end{itemize}
 %%%%%%
\section{Negotiation}
%%%%%%%%
The figure \ref{fig:2} illustrates one possible scenario of the negotiation between batch and runtime scheduler. Following points describe it in detail:
\begin{itemize}
\item The alphabets \textbf{A,B,C,D,E} represent runtime state (running jobs and idle nodes) of the partition. \textbf{TRF} means transformation of the running jobs through expand and shrink operations. This happens when a list of forwarded job requests are received from batch scheduler and the runtime scheduler runs its algorithm (based on DBES) for expanding or shrinking running jobs in order to fit as many batch jobs as possible.
\item \textbf{INIT} means initialize state. It will restore the transformed jobs back to their original state which they were at the beginning of the transaction (set of negotiations). This is done in order to perform a new transformation from the original state during every negotiation attempt otherwise the transformation will reach a saturation very soon.
\item Saturation means that we can no longer perform any significant expand or shrink operations on the running jobs as they will already be either be at their min or max node counts due to a result of the previous transformations. This would result in making no progress.
\item The green boxes labelled \textbf{"Algo 1"} represent the batch scheduling algorithm which will run every time the batch scheduler receives a resource offer. On every such attempt, it will relax the node count constraints of those jobs which could not be mapped due to lack of sufficient resources.
\item The box labelled \textbf{"Update"} will update the details of jobs in its queue once it receives a feedback. This is important as a subsequent negotiation attempt must not result in the batch scheduling algorithm dispatching a job that is already running. The feedback is the only way for the batch scheduler to recognize if the runtime scheduler accepted its mapping and started any of the jobs from the mapping it had sent.  
\item Once the negotiation has completed, the runtime scheduler will then accept the mapping and commit the jobs. The box labelled \textbf{"COMM"} represents the step where the commit happens. The forwarded jobs are added to the list of running jobs and would be started very soon. The runnings jobs will now take up their transformed state going forward as shown by the box \textbf{E}.
\end{itemize}
\begin{figure}[!b]
%\centering
\vspace{-0.50in}
\hspace*{-0.5in}
\includegraphics[width=1.2\textwidth, height=95mm]{./figures/negotiation.pdf}
\caption{Negotiation protocol}
\label{fig:2}
\end{figure}
