\chapter{\abstractname}
Invasive Computing is a novel paradigm for the design and resource-aware programming of future parallel computing systems. It enables the programmer to write resource aware programs and the goal is to optimize the program for the available resources. Traditionally, parallel applications implemented using MPI are submitted with a fixed number of MPI processes to execute on a HPC (High Performance Computing) system. This results in a fixed allocation of resources for the job. Modern techniques in scientific computing such as AMR (Adaptive Mesh Refinement) result in applications exhibiting complex behaviors where their resource requirements change during execution. Invasive MPI is an ongoing research effort to provide MPI extensions for the development of Invasive MPI applications that will result in jobs which are resource-aware for the HPC systems and can utilize such AMR techniques. Unfortunately, using only static allocations result in these applications being forced to execute using their maximum resource requirements that may lead to inefficient resource utilization. In order to support such kind of parallel applications at HPC centers, there is an urgent need to investigate and implement extensions to existing resource management systems or develop a new system. This thesis has extended the previous work during which a protocol was developed for the intgeration of invasive resource management into existing batch systems. In this work, we have explored the idea of separating the concerns of batch and runtime scheduling into two different software layers / components in contrasts to existing systems where both are merged together. Specifically, This thesis has investigated and implemented a job scheduling algorithm in accordance with the new protocol developed earlier for supporting such an invasive resource management. An early prototype that can simulate the negotiation between batch and runtime scheduler using their respective scheduling algorithms for a HPC workload comprising different job types has been accomplished.\par

%TODO: Abstract


